%% Thanks to Bennet Goeckner for giving me his TeX template, which
%% this is based on.  These percent symbols tell the compiler to
%% ignore the remainder of a given line.  We use them to write
%% comments that will not appear in the finalized output.

%% The following tells the compiler which type of document we're making.
%% There are many options. ``Article'' is probably fine for our class.
\documentclass[12pt]{article}

%% After declaring the documentclass, we load some packages which give
%% us some built-in commands and more functionality.  The following is
%% a list of packages that this file might use.  If a command you're
%% using isn't working, try Googling it -- you might need to add a
%% specific package.  I have included the standard ones that I like to
%% load.
\usepackage{amsmath}
\usepackage{amsthm}
\usepackage{amsfonts}
\usepackage{amssymb}
\usepackage{enumerate}
\usepackage{graphicx}
\usepackage{mdframed}
\usepackage{multicol}
\usepackage{verbatim}
\usepackage{tikz}
\usepackage[margin = .8in]{geometry}
\geometry{letterpaper}
\linespread{1.2}

%% One of the nicest things about LaTeX is you can create custom
%% macros. If there is a long-ish expression that you will write
%% often, it is nice to give it a shorter command.  For our common
%% number systems.
\newcommand{\RR}{\mathbb{R}} %% The blackboard-bold R that you have seen used for real numbers is typeset by $\mathbb{R}$. This macro means that $\RR$ will yield the same result, and is much shorter to type.
\newcommand{\NN}{\mathbb{N}}
\newcommand{\ZZ}{\mathbb{Z}} 
\newcommand{\QQ}{\mathbb{Q}}

%% Your macros can even accept arguments. 
\newcommand\set[1]{\left\lbrace #1 \right\rbrace} %% In mathmode, if you write \set{STUFF}, then this will output {STUFF}, i.e. STUFF inside of a set
\newcommand\abs[1]{\left| #1 \right|} %% This will do the same but with vertical bars. I.e., \abs{STUFF} gives |STUFF|
\newcommand\parens[1]{\left( #1 \right)} %% Similar. \parens{STUFF} gives (STUFF)
\newcommand\brac[1]{\left[ #1 \right]} %% Similar. \brac{STUFF} gives [STUFF]
\newcommand\sol[1]{\begin{mdframed}
\emph{Solution.} #1
\end{mdframed}}
\newcommand\solproof[1]{\begin{mdframed}
\begin{proof} #1
\end{proof}
\end{mdframed}}
%% A few more important commands:

%% You should start every proof with \begin{proof} and end it with \end{proof}.  
%%
%% Code inside single dollar signs will give in-line mathmode. I.e., $f(x) = x^2$ 
%% Code \[ \] will give mathmode centered on its own line.
%%
%% Other common commands:
%%	\begin{align*} and \end{align*} -- Good for multiline equations
%%	\begin{align} and \end{align} -- Same as above, but it will number the equations for easy reference
%%	\emph{italicized text here} and \textbf{bold text here} are also useful.
%%
%% Some very specific mathmode commands and their meanings:
%%	x \in A -- x is an element of A
%%	x \notin A -- x is not an element of A
%%	A \subseteq B -- A is a subset of B
%%	A \subsetneq B -- A is a proper subset of B
%%	x \equiv y \pmod{n} -- x is congruent to y mod n. 
%%	x \geq y and x \leq y -- Greater than or equal to and less than or equal to 
%%
%% You'll probably find lots of relevant commands in the question prompts. Also Google is your friend!

\title{Exercise Solutions for George E. Andrews' \\
  {\em Number Theory} \\
Section 1-1}

\author{Hank Alme}
  

\begin{document}
\maketitle

\begin{enumerate}
\item Prove that
  \[
  1^2 + 2^2 + 3^2 + \ldots + n^2 = \frac{n(n+1)(2n+1)}{6}
  \]

  \sol{Induction on $n$. For $n =1$, we have $1^2 = \frac{1(2)(3)}{6} = 1$. Assume then the proposition is true for $n \le k$. Adding $(k+1)^2$ to the right hand side of the proposition:
\begin{align*}
   &\frac{k(k+1)(2k+1)}{6} + (k+1)^2 \\
  =\ &\frac{(k+1)}{6} [ k(2k+1) + 6(k+1) ] \\
  =\ &\frac{(k+1)}{6} [ 2k^2+k + 6k+6 ]\\
  =\ &\frac{(k+1)}{6} [ 2k^2+7k +6 ]\\
  =\ &\frac{(k+1)}{6} [ (k+2)(2k+3) ]\\
  =\ &\frac{(k+1)}{6} [ (k+2)(2(k+1)+1) ]\\
  =\ &\frac{(k+1)(k+2)(2(k+1)+1)}{6}  
\end{align*}
the result we seek.
}

\item Prove that
  \[
  1^3 + 2^3 + 3^3 + \ldots + n^3 = (1+2+3+\ldots+n)^2
  \]
    [Hint: Use Theorem 1-1]

    \sol{Base case $n=1$ gives $1^3 = 1^2$. Assume true for $n \le
      k$. Now add $(k+1)^3$ to the right hand side:
      \begin{align*}
         & (1+2+3+\ldots+k)^2 + (k+1)^3 \\
        =\ & \left (\frac{k(k+1)}{2} \right)^2 + (k+1)^3  \\
        =\ & \frac{(k+1)^2}{4} [k^2 + 4(k+1)] \\
        =\ & \frac{(k+1)^2}{4} (k^2 + 4k+4) \\
        =\ & \frac{(k+1)^2}{4} (k+2)^2 \\
        =\ & \left (\frac{(k+1)(k+2)}{2} \right)^2 \\
        =\ & (1+2+3+\ldots+k+(k+1))^2 
      \end{align*}
      The result we seek.}


  \item Prove that
    \[
    (x^n - y^n) = (x-y)(x^{n-1} + x^{n-2}y + \ldots +xy^{n-2}+y^{n-1})
    \]

    \sol{
      The base cases are $(x-y) = (x-y)$ for $n=1$ and $x^2 - y^2 = (x-y)(x+y)$ for $n=2$ using junior high algebra's difference of squares factorization.
      
      Then, we can obeserve that for $n+1$
      \begin{align*}
        x^{n+1} - y^{n+1} =\ & x^{n+1} - x^ny + x^ny - y^{n+1} \\
        =\ & x^n(x-y) + y(x^n-y^n)
      \end{align*}
      So we can then assume that the proposition is true for $n \le k$, and compute $x^{k+1}-y^{k+1}$ using the result for $x^k-y^k$
      \begin{align*}
       x^{k+1} - y^{k+1} =\ & x^k(x-y) + y(x^k-y^k) \\
        =\ & x^k(x-y) + y(x-y)(x^{k-1}+x^{k-2}y +\ldots+xy^{k-1} + y^{k-1}) \\
        =\ & (x-y) [x^k + y(x^{k-1}+x^{k-2}y +\ldots+xy^{k-2} + y^{k-1})] \\
        =\ & (x-y) (x^k + x^{k-1}y+x^{k-2}y^2 +\ldots+xy^{k-1} + y^{k}) 
      \end{align*}
      The result we seek.
    }

  \item Prove that
    \[
    1\cdot2 + 2\cdot3 + 3\cdot4 + \ldots+n(n+1) = \frac{n(n+1)(n+2)}{3}
    \]
    \sol{Base case $n=1$ is $1\cdot2 = \frac{1(2)(3)}{6} = 2$. Assume true for $n \le k$. Adding the next term $(n+1)(n+2)$ to both sides gives
      \begin{align*}
        & \frac{n(n+1)(n+2)}{3} + (n+1)(n+2) &= (n+1)(n+2)\left(\frac{n}{3} + 1\right) \\
        =\ & (n+1)(n+2) \left(\frac{1}{3} (n+3)\right) \\
        =\ & \frac{(n+1)(n+2)(n+3)}{3}
      \end{align*}
      The result we seek.}

  \item Prove that
    \[
    1+3+5+\ldots+(2n-1) = n^2
    \]
    \sol{Base case $n=1$ gives $1 = 1^2$. Assume true for $n \le k$. Then we ad the next term $2(k+1)-1 = 2k+1$ to both sides to get
      \[
      k^2 + (2k+1) = (k+1)^2
      \]
      The result we seek.
    }

  \item Prove that
    \[
    \frac{1}{2\cdot1} + \frac{1}{2\cdot3} + \frac{1}{3\cdot4} +\ldots+\frac{1}{n(n+1)} = \frac{n}{n+1}    
    \]
    \sol{Base case $n=1$ gives $\frac{1}{1\cdot 2} = \frac{1}{2}$. Assume true for $n \le k$. Adding the next term $\frac{1}{(k+1)(k+2)}$ to both sides gives
        \begin{align*}
          & \frac{k}{k+1} + \frac{1}{(k+1)(k+2)} &= \frac{1}{k+1} \left ( k + \frac{1}{k+2} \right) \\
          =\ & \frac{1}{k+1} \left ( \frac{k^2 + 2k + 1}{k+2} \right) \\
          =\ & \frac{1}{k+1} \left ( \frac{(k+ 1)^2}{k+2} \right) \\
          =\ & \frac{k+ 1}{k+2}
        \end{align*}
        The result we seek.
    }

  \item Suppose that $F_1=1, F_2=1,F_3=2,F_4=3,F_5=5,$and in general$F_n = F_{n-1}+F_{n-2}$ for $n \ge 3$. ($F_n$ is called the $n$th Fibonacci number.) Prove that
    \[
    F_1 + F_2 + F_3 + \ldots + F_n = F_{n+2} - 1
    \]
    \sol{Base case is $n=1$. $F_3 - 1 = 2 - 1 = 1 = F_1$. Assume proposition holds for $n\le k$. Adding $F_{n+1}$ to each side
      \begin{align*}
        F_{k+2} - 1 + F_{k+1} &= F_{k+1} + F_{k+2} -1 \\
        &= F_{k+3} - 1
      \end{align*}
      The result we seek.
        
    }

  \item Prove that
    \[
    F_1 + F_3 + F_5 + \ldots + F_{2n-1} = F_{2n}
    \]
    \sol{
      Base case is $n=1$: $F_1 = 1 = F_2$. Assume proposition is true for $n \le k$. Adding the next term $F_{2(k+1)-1} = F_{2k+1}$ to the result for k gives
      \begin{align*}
        & F_{2k} + F_{2k+1} &= F_{2k+2} \\
        =\ & F_{2(k+1)}
      \end{align*}
      The result we seek.
    }

  \item Prove that
    \[
    F_2 + F_4 + F_6 + \ldots + F_{2n} = F_{2n+1} - 1
    \]
    \sol{
            Base case is $n=1$: $F_2 = 1 = 2 - 1 = F_3 - 1$. Assume proposition is true for $n \le k$. Adding the next term $F_{2(k+1)} = F_{2k+2}$ to the result for k gives
      \begin{align*}
        F_{2k+1} + F_{2k+2} - 1 &= F_{2k+3} - 1\\
        &= F_{2(k+1)+1} - 1
      \end{align*}
      The result we seek.
    }

  \item Prove that
    \[
    F_{n+1}^2 - F_nF_{n+2} = (-1)^n
    \]
    \sol{ Base case is $n=1$: $F_2^2 - F_1F_3 = 1^2 - 1\cdot2 = -1 = (-1)^1$.
      Assume proposition true for $n \le k$. Bumping the indices $k \Rightarrow k+1$ gives
      \begin{align*}
        & F_{k+2}^2 - F_{k+1}F_{k+3} &= F_{k+2}(F_k + F_{k+1}) - F_{k+1}(F_{k+1} + F_{k+2}) \\
        =\ & F_kF_{k+2} + F_{k+1}F_{k+2} -F_{k+1}^2 - F_{k+1}F_{k+2} \\
        =\ & F_kF_{k+2} - F_{k+1}^2 \\
        =\ & -(F_{k+1}^2 - F_kF_{k+2}) \\
        =\ & -((-1)^k) \\
        =\ & (-1)^{k+1}
      \end{align*}
      the result we seek.
    }

  \item Prove that
    \[
    F_1F_2 + F_2F_3 + F_3F_4 + \ldots + F_{2n-1}F_{2n} = F^2_{2n}
    \]
    \sol{ Base case is $n=1$:$F_1F_2 = 1\cdot1 = 1 = F_2^2$. Assume
      true for $n \le k$. Each bump in $n$ adds two new terms to the
      sequence. Adding the next two terms $F_{2k}F_{2k+1} +
      F_{2k+1}F_{2k+2}$ to the right hand side of the result for $n=k$
      gives
      \begin{align*}
        F^2_{2k} +F_{2k}F_{2k+1} + F_{2k+1}F_{2k+2} &= F^2_{2k} +F_{2k}F_{2k+1} + F_{2k+1}( F_{2k} + F_{2k+1}) \\
                &= F^2_{2k} + F_{2k}F_{2k+1} + F_{2k}F_{2k+1} + F_{2k+1}^2 \\
                &= F^2_{2k} + 2F_{2k}F_{2k+1} + F_{2k+1}^2 \\
        &= (F_{2k} + F_{2k+1})^2 \\
        &= F_{2k+2}^2
      \end{align*}
    }
    the result we seek.
    
  \item Prove that
    \[
    F_1F_2 + F_2F_3 + F_3F_4 + \ldots + F_{2n}F_{2n+1} = F^2_{2n+1}-1
    \]
    \sol{ Base case is $n=1$: $F_1F_2 + F_2F_3= 1\cdot1 + 1\cdot2 = 3 = 2^2 - 1 = F_3^2-1$. Assume true for $n \le k$.  Going from $k$ to $k+1$ adds two new terms $F_{2k+1}F_{2k+2}$ and $F_{2k+2}F_{2k+3}$. Looking at the right hand side of the proposition}
    \begin{align*}
      F^2_{2k+1} - 1 + F_{2k+1}F_{2k+2} + F_{2k+2}F_{2k+3} &= F^2_{2k+1} + F_{2k+1}F_{2k+2} + F_{2k+2}(F_{2k+1} + F_{2k+2}) - 1\\
      &= F^2_{2k+1} + 2F_{2k+1}F_{2k+2} +  F_{2k+2}^2 - 1\\
      &= (F_{2k+1} + F_{2k+2})^2 - 1\\
      &= F^2_{2k+3} - 1 \\
      &= F^2_{2(k+1)+1} - 1
    \end{align*}
    the result we seek.

  \item \label{ex13} The Lucas numbers $L_n$ are defined by the equations $L_1=1$, and $L_n = F_{n+1} + F_{n-1}$, for each $n \ge 2$. Prove that
    \[
    L_n = L_{n-1} + L_{n-2} \, (n \ge 3) 
    \]
    \sol{
      For the base case, consider $n=3$: using the original definition of Lucas numbers gives $L_3 = F_4 + F_2 = 3 + 1 = 4$ and the proposition gives $L_3 = L_2 + L_1 = F_3 + F_1 + L_1 = 2 + 1+1 =4$, confirming the proposition for $n=3$. Assume the proposition holds for $n \le k$, so $L_k =L_{k-1} + L_{k-2}$. Consider the right hand side of the poposition, with the indices increased by one
      \begin{align*}
        L_{k} + L_{k-1} &= F_{k+1} + F_{k-1} + F_{k} + F_{k-2} \\
        &= (F_{k} + F_{k+1}) + (F_{k-2} + F_{k-1}) \\
        &= F_{k+2} + F_k \\
        &= L_{k+1} && \text{(by definition of Lucas numbers)}
      \end{align*}
      thus proving the proposition.
      }

  \item What is wrong with the following argument?

    ``Assuming $L_n = F_n$ for $n=1,2,\ldots,k$, we see that
    \begin{align*}
      L_{k+1} &= L_k - L_{k-1} && \text{(by exercise \ref{ex13})} \\
      &= F_K + F_{k=1} && \text{(by our assumption)} \\
      &= F_{k+1} && \text{(by definition of $F_{k+1}$)}
    \end{align*}
    Hence, by the principle of mathematical induction, $F_n = L_n$ for
    each positive integer $n$.''  \sol{ The initial assumption
      (```Assuming $L_n = F_n$ for $n=1,2,\ldots,k$\ldots) is true for
      $k \le 1$, as $L_1 = F_1 = 1$, but the result from exercise
      \ref{ex13} requires $k \le 3$. If we look at $k=2$, we see $L_2
      = F_1 + F_3 = 1+2 = 3$ using the definition of Lucas numbers,
      while $F_2=1 \ne L_2$, and for $k=3$, $L_3 = F_4 + F_2 = 3 + 1 =
      4$ while $F_3 = F_2 + F_1 = 1+1 =2 \ne L_3$. The principle of
      mathematical induction does not apply here, as its preconditions
      have not been met.  }
    
  \item Prove that $F_{2n} = F_nL_n$
    \sol{      
    Base case $n=1$: $F_1L_1 = 1\cdot1 = 1 = F_2$. Assume true for $n
    \le k$. For the case $k+1$, the right hand of the proposition
    becomes
    \begin{align*}
      F_{k+1}L_{k+1} &= (F_k + F_{k-1})(F_{k+2} + F_k) \\
      &= F_kF_{k+2} + F_k^2 + F_{k-1}F_{k+2} + F_{k-1}F_k \\
      &=
    \end{align*}
    }
    
  \item Prove that
    \[
    L_1 + 2L_2 + 4L_3 + 8L_4 + \ldots + 2^{n-1}L_n = 2^nF_{n+1}-1
    \]
    \sol{
      For $n=1$, $2F_2 - 1 = 2\cdot1 - 1 = L_1$. Assume true for $n \le k$. Adding the $k+1$st term to the right hand side of the proposition gives
      \begin{align*}
        2^kF_{k+1} - 1 + 2^kL_{k+1} &= 2^k(F_{k+1} + F_{k+2} + F_{k}) -1 \\
        &= 2^k(F_{k+2} +(F_k + F_{k+1}))) - 1 \\
        &= 2^k(F_{k+2} +F_{k+2}) - 1 \\
        &= 2^k(2F_{k+2}) - 1 \\
        &= 2^{k+1}F_{k+2} - 1
      \end{align*}
      proving the proposition.
      }
    
  \item Prove that $n(n^2 -1)(3n+2)$ is divisible by $24$ for each
    positive integer $n$.
    \sol{For $n=1$, we have $1(1^2 - 1)(3\cdot1 + 2) = 0 = 24\cdot0$.
      Assume true for $n \le k$. For $k+1$ we have
      \begin{align*}
           & (k+1)((k+1)^2 - 1)(3(k+1) + 2) \\
        =\ & (k+1)(k^2 + 2k)(3k + 5) \\
        =\ & k(k+1)(k+2)(3k+5)         
      \end{align*}
      The first three factors ensure that the result is divisible by
      $3$, leaving a factor of 8 that must also divide the result.
      There are two cases, $k$ even and $k$ odd.

      First consider even $k$: $k = 2m, m >0 $. The expression becomes
      \begin{align*}
          & 2m(2m+1)(2m+2)(6m+5) \\
        =\ & 4m(2m+1)(m+1)(6m+5)
      \end{align*}
      which has a factor of $4$. To find another factor of two,
      consider the parity of $m$. If $m$ is even, then $m=2i$ for some
      $i>0$, makeing the first factor $8i$, giving us the factor we
      need. If $m$ is odd, then the factor $(m+1)$ is even, giving the
      last factor of $2$ in a similar manner. Thus if $k$ is even, the
      proposition expression has factors $3$ and $8$ and is divisible
      by 24.

      Now if $k$ is odd, $k = 2m-1, m>0$, and the $k+1$ expression becomes
      \begin{align*}
        & (2m-1)(2m)(2m+1)(3(2m-1) + 5) \\
        =\ & (2m-1)(2m)(2m+1)(6m+2) \\
        =\ & 4m(2m-1)(2m+1)(3m+1)
      \end{align*}
      with a factor of $4$. Getting the final factor of 2 is simlar:
      if m is even, then $4m = 4\cdot2i = 8i$ for some $i>0$, and if m
      is odd then $m = 2i-1$ for some $i > 0$ and $(3m+1) = (3(2i-1) +
      1) = (6i-2) = 2(3i-1)$, giving the needed factor.
      
      Since in both cases ($k$ even and $k$ odd) the expression for
      $k+1$ is divisible by $24$, the proposition is proven.  }

    \item Prove that if $n$ is an odd positive integer, then $x+y$ is
      a factor of $x^n+y^n$. (For example, if $n=3$,then $x^3+y^3 =
      (x+y)(x^2 - xy + y^2)$.)  \sol{ For $n=1$ we get the trivial
        result $x+y$ which obviously meets requirements. Assume the
        poposition true for $n \in \{1,3,5,...,2k-1\}$, so $x^{2k-1} +
        y^{2k-1} = (x+y)f(x,y)$ for some $f(x,y)$. Consider the next
        odd exponent $2k+1$ which gives $x^{2k+1} + y^{2k+1}$:
        \begin{align*}
          =\ & x^{2k+1} + +x^2y^{2k-1} - x^2y^{2k-1}k + y^{2k+1} \\
          =\ & x^2(x^{2k-1} + y^{2k-1}) + y^{2k-1}(y^2 - x^2) \\
          =\ & x^2(x+y)f(x,y) - y^2(x^2-y^2) \\
          =\ & x^2(x+y)f(x,y) - y^2(x+y)(x-y) \\
          =\ & (x+y)\left [ x^2f(x,y) - y^2(x-y) \right ]
          \end{align*}
      }
      proving the proposition.
    
\end{enumerate}
\end{document}
 
