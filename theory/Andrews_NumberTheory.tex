%% Thanks to Bennet Goeckner for giving me his TeX template, which
%% this is based on.  These percent symbols tell the compiler to
%% ignore the remainder of a given line.  We use them to write
%% comments that will not appear in the finalized output.

%% The following tells the compiler which type of document we're making.
%% There are many options. ``Article'' is probably fine for our class.
\documentclass[12pt]{article}

%% After declaring the documentclass, we load some packages which give
%% us some built-in commands and more functionality.  The following is
%% a list of packages that this file might use.  If a command you're
%% using isn't working, try Googling it -- you might need to add a
%% specific package.  I have included the standard ones that I like to
%% load.
\usepackage{amsmath}
\usepackage{amsthm}
\usepackage{amsfonts}
\usepackage{amssymb}
\usepackage{enumerate}
\usepackage{graphicx}
\usepackage{mdframed}
\usepackage{multicol}
\usepackage{verbatim}
\usepackage{tikz}
\usepackage[margin = .8in]{geometry}
\geometry{letterpaper}
\linespread{1.2}

%% One of the nicest things about LaTeX is you can create custom
%% macros. If there is a long-ish expression that you will write
%% often, it is nice to give it a shorter command.  For our common
%% number systems.
\newcommand{\RR}{\mathbb{R}} %% The blackboard-bold R that you have seen used for real numbers is typeset by $\mathbb{R}$. This macro means that $\RR$ will yield the same result, and is much shorter to type.
\newcommand{\NN}{\mathbb{N}}
\newcommand{\ZZ}{\mathbb{Z}} 
\newcommand{\QQ}{\mathbb{Q}}

%% Your macros can even accept arguments. 
\newcommand\set[1]{\left\lbrace #1 \right\rbrace} %% In mathmode, if you write \set{STUFF}, then this will output {STUFF}, i.e. STUFF inside of a set
\newcommand\abs[1]{\left| #1 \right|} %% This will do the same but with vertical bars. I.e., \abs{STUFF} gives |STUFF|
\newcommand\parens[1]{\left( #1 \right)} %% Similar. \parens{STUFF} gives (STUFF)
\newcommand\brac[1]{\left[ #1 \right]} %% Similar. \brac{STUFF} gives [STUFF]
\newcommand\sol[1]{\begin{mdframed}
\emph{Solution.} #1
\end{mdframed}}
\newcommand\solproof[1]{\begin{mdframed}
\begin{proof} #1
\end{proof}
\end{mdframed}}
%% A few more important commands:

%% You should start every proof with \begin{proof} and end it with \end{proof}.  
%%
%% Code inside single dollar signs will give in-line mathmode. I.e., $f(x) = x^2$ 
%% Code \[ \] will give mathmode centered on its own line.
%%
%% Other common commands:
%%	\begin{align*} and \end{align*} -- Good for multiline equations
%%	\begin{align} and \end{align} -- Same as above, but it will number the equations for easy reference
%%	\emph{italicized text here} and \textbf{bold text here} are also useful.
%%
%% Some very specific mathmode commands and their meanings:
%%	x \in A -- x is an element of A
%%	x \notin A -- x is not an element of A
%%	A \subseteq B -- A is a subset of B
%%	A \subsetneq B -- A is a proper subset of B
%%	x \equiv y \pmod{n} -- x is congruent to y mod n. 
%%	x \geq y and x \leq y -- Greater than or equal to and less than or equal to 
%%
%% You'll probably find lots of relevant commands in the question prompts. Also Google is your friend!

\title{Exercise Solutions for George E. Andrews' \\
  {\em Number Theory} \\
Section 1-1}

\author{Hank Alme}
  

\begin{document}
\maketitle

\begin{enumerate}
\item Prove that
  \[
  1^2 + 2^2 + 3^2 + \ldots + n^2 = \frac{n(n+1)(2n+1)}{6}
  \]

  \sol{Induction on $n$. For $n =1$, we have $1^2 = \frac{1(2)(3)}{6} = 1$. Assume then the proposition is true for $n \le k$:
    \[
1^2 + 2^2 + 3^2 + \ldots + k^2 = \frac{k(k+1)(2k+1)}{6}
\]
Adding $(k+1)^2$:
\begin{eqnarray*}
  &=&\frac{k(k+1)(2k+1)}{6} + (k+1)^2 \\
  &=&\frac{(k+1)}{6} [ k(2k+1) + 6(k+1) ] \\
  &=&\frac{(k+1)}{6} [ 2k^2+k + 6k+6 ]\\
  &=&\frac{(k+1)}{6} [ 2k^2+7k +6 ]\\
  &=&\frac{(k+1)}{6} [ (k+2)(2k+3) ]\\
  &=&\frac{(k+1)}{6} [ (k+2)(2(k+1)+1) ]\\
  &=&\frac{(k+1)(k+2)(2(k+1)+1)}{6}  
\end{eqnarray*}
the result we seek.
}

\item Prove that
  \[
  1^3 + 2^3 + 3^3 + \ldots + n^3 = (1+2+3+\ldots+n)^2
  \]
    [Hint: Use Theorem 1-1]

    \sol{Base case $n=1$ gives $1^3 = 1^2$. Assume true for $n \le
      k$. Now add $(k+1)^3$ to both sides
      \begin{eqnarray*}
        &=& (1+2+3+\ldots+k)^2 + (k+1)^3 \\
        &=& \left (\frac{k(k+1)}{2} \right)^2 + (k+1)^3  \\
        &=& \frac{(k+1)^2}{4} [k^2 + 4(k+1)] \\
        &=& \frac{(k+1)^2}{4} (k^2 + 4k+4) \\
        &=& \frac{(k+1)^2}{4} (k+2)^2 \\
        &=& \left (\frac{(k+1)(k+2)}{2} \right)^2 \\
        &=& (1+2+3+\ldots+k+(k+1))^2 
      \end{eqnarray*}
      The result we seek.}


  \item Prove that
    \[
    (x^n - y^n) = (x-y)(x^{n-1} + x^{n-2}y + \ldots +xy^{n-2}+y^{n-1})
    \]

    \sol{
      The base cases are $(x-y) = (x-y)$ for $n=1$ and $x^2 - y^2 = (x-y)(x+y)$ for $n=2$ using junior high algebra's difference of squares factorization.
      
      Then, we can obeserve that for $n+1$
      \begin{eqnarray*}
        x^{n+1} - y^{n+1} &=& x^{n+1} - x^ny + x^ny - y^{n+1} \\
        &=& x^n(x-y) + y(x^n-y^n)
      \end{eqnarray*}
      So we can then assume that the proposition is true for $n \le k$, and compute $x^{k+1}-y^{k+1}$ using the result for $x^k-y^k$
      \begin{eqnarray*}
        x^{k+1} - y^{k+1} &=& x^k(x-y) + y(x^k-y^k) \\
        &=& x^k(x-y) + y(x-y)(x^{k-1}+x^{k-2}y +\ldots+xy^{k-1} + y^{k-1}) \\
        &=& (x-1) [x^k + y(x^{k-1}+x^{k-2}y +\ldots+xy^{k-2} + y^{k-1})] \\
        &=& (x-1) (x^k + x^{k-1}y+x^{k-2}y^2 +\ldots+xy^{k-1} + y^{k}) 
      \end{eqnarray*}
      The result we seek.
    }

  \item Prove that
    \[
    1\cdot2 + 2\cdot3 + 3\cdot4 + \ldots+n(n+1) = \frac{n(n+1)(n+2)}{3}
    \]
    \sol{Base case $n=1$ is $1\cdot2 = \frac{1(2)(3)}{6} = 2$. Assume true for $n \le k$. Adding the next term $(n+1)(n+2)$ to both sides gives
      \begin{eqnarray*}
        \frac{n(n+1)(n+2)}{3} + (n+1)(n+2) &=& (n+1)(n+2)\left(\frac{n}{3} + 1\right) \\
        &=& (n+1)(n+2) \left(\frac{1}{3} (n+3)\right) \\
        &=& \frac{(n+1)(n+2)(n+3)}{3}
      \end{eqnarray*}
      The result we seek.}

  \item Prove that
    \[
    1+3+5+\ldots+(2n-1) = n^2
    \]
    \sol{Base case $n=1$ gives $1 = 1^2$. Assume true for $n \le k$. Then we ad the next term $2(k+1)-1 = 2k+1$ to both sides to get
      \[
      k^2 + (2k+1) = (k+1)^2
      \]
      The result we seek.
    }

  \item Prove that
    \[
    \frac{1}{2\cdot1} + \frac{1}{2\cdot3} + \frac{1}{3\cdot4} +\ldots+\frac{1}{n(n+1)} = \frac{n}{n+1}    
    \]
    \sol{Base case $n=1$ gives $\frac{1}{1\cdot 2} = \frac{1}{2}$. Assume true for $n \le k$. Adding the next term $\frac{1}{(k+1)(k+2)}$ to both sides gives
        \begin{eqnarray*}
          \frac{k}{k+1} + \frac{1}{(k+1)(k+2)} &=& \frac{1}{k+1} \left ( k + \frac{1}{k+2} \right) \\
          &=& \frac{1}{k+1} \left ( \frac{k^2 + 2k + 1}{k+2} \right) \\
          &=& \frac{1}{k+1} \left ( \frac{(k+ 1)^2}{k+2} \right) \\
          &=& \frac{k+ 1}{k+2}
        \end{eqnarray*}
        The result we seek.
    }
\end{enumerate}
\end{document}
