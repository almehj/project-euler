%% Thanks to Bennet Goeckner for giving me his TeX template, which
%% this is based on.  These percent symbols tell the compiler to
%% ignore the remainder of a given line.  We use them to write
%% comments that will not appear in the finalized output.

%% The following tells the compiler which type of document we're making.
%% There are many options. ``Article'' is probably fine for our class.
\documentclass[12pt]{article}

%% After declaring the documentclass, we load some packages which give
%% us some built-in commands and more functionality.  The following is
%% a list of packages that this file might use.  If a command you're
%% using isn't working, try Googling it -- you might need to add a
%% specific package.  I have included the standard ones that I like to
%% load.
\usepackage{amsmath}
\usepackage{amsthm}
\usepackage{amsfonts}
\usepackage{amssymb}
\usepackage{enumerate}
\usepackage{graphicx}
\usepackage{mdframed}
\usepackage{multicol}
\usepackage{verbatim}
\usepackage{tikz}
\usepackage[margin = .8in]{geometry}
\geometry{letterpaper}
\linespread{1.2}

%% One of the nicest things about LaTeX is you can create custom
%% macros. If there is a long-ish expression that you will write
%% often, it is nice to give it a shorter command.  For our common
%% number systems.
\newcommand{\RR}{\mathbb{R}} %% The blackboard-bold R that you have seen used for real numbers is typeset by $\mathbb{R}$. This macro means that $\RR$ will yield the same result, and is much shorter to type.
\newcommand{\NN}{\mathbb{N}}
\newcommand{\ZZ}{\mathbb{Z}} 
\newcommand{\QQ}{\mathbb{Q}}

%% Your macros can even accept arguments. 
\newcommand\set[1]{\left\lbrace #1 \right\rbrace} %% In mathmode, if you write \set{STUFF}, then this will output {STUFF}, i.e. STUFF inside of a set
\newcommand\abs[1]{\left| #1 \right|} %% This will do the same but with vertical bars. I.e., \abs{STUFF} gives |STUFF|
\newcommand\parens[1]{\left( #1 \right)} %% Similar. \parens{STUFF} gives (STUFF)
\newcommand\brac[1]{\left[ #1 \right]} %% Similar. \brac{STUFF} gives [STUFF]
\newcommand\sol[1]{\begin{mdframed}
\emph{Solution.} #1
\end{mdframed}}
\newcommand\solproof[1]{\begin{mdframed}
\begin{proof} #1
\end{proof}
\end{mdframed}}
%% A few more important commands:

%% You should start every proof with \begin{proof} and end it with \end{proof}.  
%%
%% Code inside single dollar signs will give in-line mathmode. I.e., $f(x) = x^2$ 
%% Code \[ \] will give mathmode centered on its own line.
%%
%% Other common commands:
%%	\begin{align*} and \end{align*} -- Good for multiline equations
%%	\begin{align} and \end{align} -- Same as above, but it will number the equations for easy reference
%%	\emph{italicized text here} and \textbf{bold text here} are also useful.
%%
%% Some very specific mathmode commands and their meanings:
%%	x \in A -- x is an element of A
%%	x \notin A -- x is not an element of A
%%	A \subseteq B -- A is a subset of B
%%	A \subsetneq B -- A is a proper subset of B
%%	x \equiv y \pmod{n} -- x is congruent to y mod n. 
%%	x \geq y and x \leq y -- Greater than or equal to and less than or equal to 
%%
%% You'll probably find lots of relevant commands in the question prompts. Also Google is your friend!

\title{Exercise Solutions for George E. Andrews' \\
  {\em Number Theory} \\
Section 1-2}

\author{Hank Alme}
  

\begin{document}
\maketitle

\begin{enumerate}
\item Write the numbers twenty-five, thirty-two, and fifty-six to the base five.

  \sol{
    \begin{align*}
      25 &= 1\cdot5^2 + 0\cdot5^1 + 0\cdot5^0 = 100\ \text{base 5} \\
      32 &= 1\cdot5^2 + 1\cdot5^1 + 2\cdot5^0 = 112\ \text{base 5} \\
      56 &= 2\cdot5^2 + 1\cdot5^1 + 1\cdot5^0 = 211\ \text{base 5}
    \end{align*}
}

  \item Write the numbers forty-seven, sixty-eight, and one hundred
    twenty-seven to the base two.
  \sol{
    \begin{align*}
      47 &= 0\cdot2^6 + 1\cdot2^5 +  0\cdot2^4 +  1\cdot2^3 +  1\cdot2^2 +  1\cdot2^1 +  1\cdot2^0  = 101111\ \text{base 2} \\
      68 &= 1\cdot2^6 +  0\cdot2^5 +  0\cdot2^4 +  0\cdot2^3 +  1\cdot2^2 +  0\cdot2^1 +  1\cdot2^0  = 1000100\ \text{base 2}\\
      127 &= 1\cdot2^6 +  1\cdot2^5 +  1\cdot2^4 +  1\cdot2^3 +  1\cdot2^2 +  1\cdot2^1 +  1\cdot2^0  = 1111111\ \text{base 2}
      \end{align*}
  }

\item What is the least number of weights required to weigh any
  integral number of pounds up to 63 pounds if one is only allowed to
  put weights on one side of the balance?

  \sol{ This requires a minimum of six weights: $32,16,8,4,2,$ and $1$
    pounds. With this set, any weight up to $63$ can be expressed
    using its binary representation, indicating which weights need to
    be included. For example, to weigh $25$ pounds ($011001$ binary),
    we would use the $16 (2^4), 8 (2^3),$ and $1 (2^0)$ pound weights.  }

\item \label{ex4} Prove that each non-zero integer may be represented in the form
  \[
  n = \sum_{j=0}^s c_j3^j
  \]
  where $c_s \ne 0$, and each $c_j$ is equal to $-1,0,$ or $1$.

\item Using Exercise \ref{ex4}, determine the least number of weights required to weigh any integral number of pounds up to 80 pounds if one is allowed to put weights in both pans of a balance.

\item \label{ex6} Prove that if
  \[
  a_sk^s + a_{s-1}k^{s-1} + \ldots + a_0
  \]
  is a representation of $n$ to the base $k$, then $0 < n \le k^{s+1}
  - 1$. [Hint: use Theorem 1-2.]

  \item Without using Theorem 1-3, prove directly that two different representations to the base $k$ represent different integers. [Hint: Use Exercise \ref{ex6}.]
\end{enumerate}
\end{document}
% 31 28 31 30 31 30 31 31 30 31 30 31
% +3 +0 +3 +2 +3 +2 +3 +3 +2 +3 +2 +3
% +3 +3 +6 +1 +4 +6 +2 +5 +0 +3 +5 +1
